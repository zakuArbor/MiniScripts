\documentclass{article}
\usepackage[utf8]{inputenc}
\usepackage{listings}

\title{C Programming}
\author{Ju Hong Kim }
\date{December 2017}

\begin{document}

\maketitle

\textbf{0.1 Motivation}
\\~\\

When I first learned programming, I always seemed to have struggled with learning and finding a good material to learn programming. With many materials available online, it is often hard to choose from. The internet is a great source of educational materials that has empowered and enabled the human civilization to get easy access to free education with a click of a button. 
\\

The option of having tons of great resources online has led to students scavenging for good materials to use. This can cause a headache so I hope that this book will help students not scavenge on the internet for materials.
\\

But this was not my source of motivation for writing this book. I have seen many students struggle in introductory programming courses and give up. Learning how to program requires you to think differently on how you would approach a problem. This can prove to be a big challenge to many who are not wired to think of how to create a though process on how to problem solve in a very systematic way. We humans naturally do things without much thinking such as walking or sorting a book in a bookshelf. I hope this book will allow students to learn and succeed in whatever course they are studying.
\\

This book is geared specifically to CSC209 Software Tools and Systems Programming (a course taught in 2nd year at the University of Toronto) but anyone is free to learn and give feedback about the book. 
\\~\\

\textbf{1 Introduction}
\\~\\

Welcome to CSC209 Software Tools and Systems Programming. This course will cover a great depth on software techniques and system tools that will prove to be useful in building effective programs and interacting with the Operating System. 

The course will typically cover Shell Scripting and C Programming to help students be able to understand what goes on in the operating system, how to create software tools and learn different ways of communicating between processes such as forks and basic networking and Unix. 
\\~\\

\textbf{1.1 Why Should I take this course?}
\\

This course will teach you three great concepts in Computer Science:

\begin{enumerate}
    \item C Programming
    \item System Programming
    \item Unix Tools
\end{enumerate}

[ASK INPUT ON TO WHY STUDENTS SHOULD TAKE THIS COURSE]

\textbf{Reasons to learn C}
\\

There are tons of reasons to learn C programming. My first reason is the connection you can have between the hardware and you the programmer. C can be consider a low level programming language, giving a great connection between you the programmer and the computer. Low level languages allows you to be able to understand how to program more efficiently by being more aware of how your program interacts with the hardware and the operating system. 
\\

"When you learn to program in C you almost have to gain an understanding of how programs execute. You know what things like register, stack, heap and memory mapped IO mean" - Tod Gentille
\\

Having an understanding of how programs execute and a basic understanding of how the operating system works is part of what we want you to get out of this course. This course is a prelude to CSC369 (Operating Systems) where you will learn "one of the greatest inventions of mankind" (Larry). 
\\

C was designed to be efficient and elegant to your computer so that it runs very fast on your computer. Unlike high level programming languages where there are a lot of abstractions to make your life easier, C is very simple and easy to compile. Which is why many students may find C programming hard at first since there is no such abstractions that you typically would see in higher level languages such as classes, lambda functions (You'll learn this in CSC324 Principles of Programming Languages or in Functional Programming Lanauges), and macros. Abstractions has a lot of benefits and makes code much more readable but has a higher performance and efficiency cost. 

\\
Another great reason to learn C is the portability of the language. C has been around for decades and so you will see a lot of devices around you that powers in C such as your computer's OS, embedded systems and Relational Database Management Systems. If a program runs in several architectures, then there is a good chance that it runs in C. Since C is one of the closest language to machine code as possible, compilers can compile C code to machine code very quickly so that you can have an executable program that your computer can run. Furthermore, a lot of Operating Systems are coded in C which comes to no surprise. In fact, the language was originally designed for system level and embedded software development due to its close ties with the UNIX Development. 

{talk about derivatives of C such as Java, C++, and etc}


-talk about python and its weak typed language

 Unix-style environment, using scripting languages and a machine-oriented programming language (typically C). What goes on in the operating system when programs are executed. Core topics: creating and using software tools, pipes and filters, file processing, shell programming, processes, system calls, signals, basic network programming.

The "software tools" model and unix philosophy, i/o redirection, some shell programming, the unix filesystem, the C programming language (syntax, datatypes, storage model), unix processes, the operating system interface, interprocess communication and network communication, introduction to concurrency, and unix and internet security. 

\section{Is This Book Necessary?}

\section{How do I do well in this course}
Be Interested

\textbf{Who should read this book?}

\section{Fundamentals of C}

\textbf{1.1 Getting Started: "Hello World"}

To start off with an introduction to C, here's a code to illustrate the very basic syntax of C and also as a sanity check to see if you have C correctly installed and compiled correctly. 

\begin{lstlisting}
#include <stdio.h>

int main()
{
	printf("hello, world\n");
    return 0;
}
\end{lstlisting}

\lstinputlisting[caption=Scheduler, style=customc]{hello.c}

To compile and run this program will depend on the system you are using. This book will assume you are using a Linux Machine such as the ones available in the Linux Labs at Deerfield Hall and the labs at Bahem Center. If you are using a Mac then it should work in the same way as the Linux Machines. I would highly recommend students using Windows to SSH into the Linux Machines to work on their assignments and Labs. 

To compile the program, use the following command: 

gcc -Wall hello.c


If there are no warnings or errors such as omitting a character or missing a semi-colon, then the compilation should suceed and make an executable file called a.out

To run the program: ./a.out

Once you run the program, the output should be:

hello, world

{Get images}


Line 1: #include <stdio.h>

Explaination: We are referencing the standard input/output library which contains code for builtin C functions related to file input and output. This will further discussed in later chapters. For now, just understand the fact that if you wish to use builtin functions, just reference the library at the beginning of your code so that your compiler knows where to find the piece of code.

Line 3: int main()







{talk about includes, preprocessors and etc on another section}
{talk about libraries in another section: https://www.programiz.com/c-programming/library-function}

{source of information should cite C modern approach to programming, C the Programming Lanaguage}
\end{document}

